% !TEX encoding = UTF-8 Unicode
\documentclass{beamer}

\usepackage{amsmath}
\usepackage{color}
\usepackage{gensymb}
\usepackage{hyperref}
\usepackage{textcomp}

\usetheme{Warsaw}

\newcommand{\btVFill}{\vskip0pt plus 1filll}

\title[Daily monitoring of low-frequency earthquake activity]{Daily monitoring of low-frequency earthquake activity}
\author{Ariane Ducellier, Scott Henderson}
\date{Winter 2020 Incubator project}

\begin{document}

	\begin{frame}
		\titlepage
	\end{frame}

	\section{The problem}
	
	\begin{frame}
		\frametitle{Low-frequency earthquakes}
		\begin{itemize}
			\item Small magnitude (M $\sim$ 1)
			\item Dominant frequency low (1-10 Hz) compared with that of ordinary tiny earthquakes (up to 20 Hz)
			\item Source located on the plate boundary
			\item Grouped into families of events, with all the earthquakes of a given family originating from the same small patch on the plate interface
			\item Recurrence more or less episodic in a bursty manner
		\end{itemize}
	\end{frame}

	\begin{frame}
		\frametitle{What we do now}
	\end{frame}

	\begin{frame}
		\frametitle{What we aim to do}
	\end{frame}

	\begin{frame}
		\frametitle{First step: Python package}
	\end{frame}

	\begin{frame}
		\frametitle{Second step: Command line}
	\end{frame}

	\begin{frame}
		\frametitle{Third step: CronJob}
	\end{frame}

	\begin{frame}
		\frametitle{Fourth step: Improving computing time}
	\end{frame}

	\begin{frame}
		\frametitle{Fifth step: Saving results}
	\end{frame}

	\begin{frame}
		\frametitle{To do in the future: Amazon Lambda}
	\end{frame}

	\begin{frame}
		\frametitle{Conclusion}
	\end{frame}

\end{document}
